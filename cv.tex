%----------------------------------------------------------------------------------------
%	DOCUMENT DEFINITION
%----------------------------------------------------------------------------------------

% article class because we want to fully customize the page and not use a cv template
\documentclass[a4paper,12pt]{article}

%----------------------------------------------------------------------------------------
%	PACKAGES
%----------------------------------------------------------------------------------------
\usepackage{url}
\usepackage{parskip} 	

%other packages for formatting
\RequirePackage{color}
\RequirePackage{graphicx}
\usepackage[usenames,dvipsnames]{xcolor}
\usepackage[scale=0.9]{geometry}

%tabularx environment
\usepackage{tabularx}

%for lists within experience section
\usepackage{enumitem}

% centered version of 'X' col. type
\newcolumntype{C}{>{\centering\arraybackslash}X} 

%to prevent spillover of tabular into next pages
\usepackage{supertabular}
\newlength{\fullcollw}
\setlength{\fullcollw}{0.47\textwidth}

%custom \section
\usepackage{titlesec}				
\usepackage{multicol}
\usepackage{multirow}

%CV Sections inspired by: 
%http://stefano.italians.nl/archives/26
\titleformat{\section}{\Large\scshape\raggedright}{}{0em}{}[\titlerule]
\titlespacing{\section}{0pt}{10pt}{10pt}

%for publications
\usepackage[style=authoryear,sorting=ynt, maxbibnames=2]{biblatex}

%Setup hyperref package, and colours for links
\usepackage[unicode, draft=false]{hyperref}
\definecolor{linkcolour}{rgb}{0,0.2,0.6}
\hypersetup{colorlinks,breaklinks,urlcolor=linkcolour,linkcolor=linkcolour}
\addbibresource{citations.bib}
\setlength\bibitemsep{1em}

%for social icons
\usepackage{fontawesome5}

% job listing environments
\newenvironment{jobshort}[2]
    {
    \begin{tabularx}{\linewidth}{@{}l X r@{}}
    \textbf{#1} & \hfill &  #2 \\[3.75pt]
    \end{tabularx}
    }
    {
    }

\newenvironment{joblong}[3]
    {
    \begin{tabularx}{\linewidth}{@{}l X r@{}}
    \textbf{#1} & \hfill & #2 \\[3.75pt]
    \end{tabularx}
    \textit{#3} \\[4pt]
    \begin{minipage}[t]{\linewidth}
    \begin{itemize}[nosep,after=\strut, leftmargin=2em, itemsep=3pt,label=--]
    }
    {
    \end{itemize}
    \end{minipage}
    }

% project listing enviroments
\newenvironment{projectlong}[2]
    {
    \begin{tabularx}{\linewidth}{@{}l X r@{}}
    \textbf{#1} & \hfill &  #2 \\[3.75pt]
    \end{tabularx} \\
    \begin{minipage}[t]{\linewidth}
    \begin{itemize}[nosep,after=\strut, leftmargin=2em, itemsep=3pt,label=--]
    }
    {
    \end{itemize}
    \end{minipage}
    }
%----------------------------------------------------------------------------------------
%	BEGIN DOCUMENT
%----------------------------------------------------------------------------------------
\begin{document}

% non-numbered pages
\pagestyle{empty} 

%----------------------------------------------------------------------------------------
%	TITLE
%----------------------------------------------------------------------------------------

% \begin{tabularx}{\linewidth}{ @{}X X@{} }
% \huge{Your Name}\vspace{2pt} & \hfill \emoji{incoming-envelope} email@email.com \\
% \raisebox{-0.05\height}\faGithub\ username \ | \
% \raisebox{-0.00\height}\faLinkedin\ username \ | \ \raisebox{-0.05\height}\faGlobe \ mysite.com  & \hfill \emoji{calling} number
% \end{tabularx}

\begin{tabularx}{\linewidth}{@{} C @{}}
\Huge{Nicole Emilia Tineo Gonzales} \\[7.5pt]
\href{https://github.com/nicole-tineo}{\raisebox{-0.05\height}\faGithub\ nicole-tineo} \textbar\ 
\href{https://www.linkedin.com/in/nicole-tineo/}{\raisebox{-0.05\height}\faLinkedin\ Nicole Emilia Tineo Gonzales} \textbar\ 
\href{mailto:n.tineogonzales@gmail.com}{\raisebox{-0.05\height}\faEnvelope \ n.tineogonzales@gmail.com} \textbar\ 
\href{tel:+51970147320}{\raisebox{-0.05\height}\faMobile \ 970147320}
\end{tabularx}


%----------------------------------------------------------------------------------------
% EXPERIENCE SECTIONS
%----------------------------------------------------------------------------------------

%Interests/ Keywords/ Summary
\section{Resumen}
Ingeniera Industrial especializada en gestión de proyectos de mejora y optimización de procesos en
entornos educativos y de servicios. Experiencia liderando implementaciones operativas, rediseñando
flujos críticos y tomando decisiones bajo restricciones de tiempo, presupuesto y recursos. He intervenido
procesos académicos, comerciales y administrativos con enfoque en cumplimiento de plazos, eficiencia
operativa y experiencia del usuario, utilizando datos para priorizar, controlar y sostener mejoras.

%Experience
\section{Experiencia Laboral}

\begin{joblong}{Analista de Operaciones y Proyectos Académicos}{Jul 2025 - Presente}{Universidad Tecnológica del Perú}
    \item Lidero la implementación operativa de aulas y laboratorios para programas académicos, priorizando requerimientos
    críticos y gestionando restricciones de stock, tiempos y proveedores, asegurando el inicio de clases sin incidencias.
    \item Tomo decisiones de priorización de recursos en función de impacto académico y urgencia operativa, coordinando con
    Direcciones Académicas, Logística y Campus.
    \item Elaboro cronogramas, matrices de seguimiento y reportes de avance para el control de hitos, desviaciones y
    cumplimiento de plazos.
    \item Identifico y ejecuto mejoras en procesos académicos relacionados a inventarios, reposiciones y atención interna,
    optimizando tiempos de respuesta y uso de recursos.
\end{joblong}

\begin{joblong}{Asistente de Dirección (Temporal)}{Abr 2025 – Jul 2025}{Universidad de Ingeniería y Tecnología}
    \item Gestioné procesos académicos y administrativos en carreras de Ingeniería y Marketing, interviniendo flujos internos
    para reducir fricciones en la gestión docente y atención al estudiante.
    \item Administré presupuestos OPEX/CAPEX y el ciclo completo de pagos mediante NetSuite (Oracle), asegurando trazabilidad
    y cumplimiento financiero.
    \item Rediseñé flujos de solicitudes académicas mediante automatizaciones en Excel, reduciendo reprocesos y tiempos de atención.
    \item Analicé procesos AS IS en plataformas académicas (Canvas, Core), proponiendo mejoras en matrícula y carga académica con
    enfoque en experiencia del usuario.
\end{joblong}

\begin{joblong}{Asistente y Analista Junior Comercial}{Jun 2023 – Feb 2025}{Interseguro Compañía de Seguros}
    \item Analicé datos comerciales y construí dashboards en Power BI y Looker Studio para monitorear desempeño de ventas y eficiencia
    de asignación de casos.
    \item Intervine procesos de distribución de oportunidades comerciales, reduciendo tiempos de respuesta y mejorando la trazabilidad
    por región y asesor.
    \item Diseñé y documenté flujos operativos para pagos, órdenes de compra y provisión de materiales, estandarizando procesos
    repetitivos.
    \item Implementé controles presupuestales y alertas automatizadas en Excel avanzado para seguimiento continuo del gasto comercial.
\end{joblong}

\begin{joblong}{Practicante de Administración y Finanzas}{2022}{Corporación Trazzo Inmobiliaria}
    \item Gestioné pagos, facturación y control de flujos de caja, asegurando orden financiero en proyectos inmobiliarios.
    \item Implementé controles de órdenes de compra y seguimiento de gastos por proyecto, mejorando la visibilidad financiera mensual.
\end{joblong}

%Projects
\section{Proyectos}
\begin{projectlong}{Optimización de Gestión Financiera en Proyectos Inmobiliarios}{2025}
    \item Diseñé un sistema de control en Excel para la trazabilidad de gastos con tarjetas corporativas, mejorando el cierre contable y reduciendo errores financieros.
    \item La herramienta permitió análisis estructurado del gasto y soporte directo a la toma de decisiones.
\end{projectlong}

\begin{projectlong}{Dashboard de Seguimiento de Casos Comerciales}{2024}
    \item Desarrollé un dashboard en Looker Studio para seguimiento en tiempo real de casos referidos, permitiendo identificar brechas de desempeño por asesor y jefatura.
    \item La solución fortaleció la toma de decisiones comerciales y el control de indicadores clave.
\end{projectlong}

\begin{projectlong}{Sistema de Gestión y Trazabilidad de Solicitudes Operativas Académicas (UTP)}{2025}
    \item Diseñé e implementé un sistema centralizado de gestión de solicitudes para equipos, mobiliario, herramientas e insumos académicos, utilizando Excel en SharePoint
    con formularios automatizados mediante macros, permitiendo el registro en línea y estandarizado de requerimientos.
    \item Eliminé la gestión informal de solicitudes (correos y mensajes dispersos), mejorando la trazabilidad, control y priorización de requerimientos por tipo, programa
    académico y estado.
    \item Integré el sistema con Power Automate, automatizando flujos de notificación y seguimiento mediante envío de correos automáticos a través de un bot, reduciendo
    tiempos de respuesta y reprocesos operativos.
    \item Generé una base de datos estructurada que permite seguimiento en tiempo real, control de pendientes y soporte a la toma de decisiones sobre asignación de recursos.
    \item El sistema fortaleció el control operativo previo al inicio de clases, reduciendo riesgos de quiebres operativos y fallas por falta de visibilidad de solicitudes.
\end{projectlong}

%----------------------------------------------------------------------------------------
%	EDUCATION
%----------------------------------------------------------------------------------------
\section{Educación}
\begin{tabularx}{\linewidth}{@{}l X@{}}	
    2018 - 2023 & Ingeniería Industrial - \textbf{Universidad Ricardo Palma} \\
    2025 & Programa de Especialización en Gestión por Procesos y Mejora Continua - \textbf{Centrum PUCP} \\
    2025 & Programa de Especialización en Gerencia de Proyectos (PMBOK 7) - \textbf{Universidad Nacional de Ingeniería} \\
    2021 & Inglés Avanzado - \textbf{ICPNA} \\
\end{tabularx}

%----------------------------------------------------------------------------------------
%	PUBLICATIONS
%----------------------------------------------------------------------------------------
% \section{Publications}
% \begin{refsection}[citations.bib]
% \nocite{*}
% \printbibliography[heading=none]
% \end{refsection}

%----------------------------------------------------------------------------------------
%	SKILLS
%----------------------------------------------------------------------------------------
\section{Habilidades}

\begin{tabularx}{\linewidth}{@{}l X@{}}
    \textbf{Lenguajes} & SQL \\[3pt]
    \textbf{Data \& BI} & Power BI, Looker Studio, Excel \\[3pt]
    \textbf{Automatización} & Power Automate, Power Platform, VBA/macros\\[3pt]
    \textbf{Bases de Datos} & BigQuery \\[3pt]
    \textbf{Control de Versiones} & Git \\[3pt]
    \textbf{Habilidades Blandas} & Pensamiento analítico, resolución de problemas, orientación a resultados, organización y planificación, trabajo colaborativo, enfoque en el usuario \\
\end{tabularx}

%----------------------------------------------------------------------------------------
%	Languajes
%----------------------------------------------------------------------------------------

\section{Lenguajes}

\begin{tabularx}{\linewidth}{@{}l X@{}}
    \textbf{Inglés} & Avanzado \\
\end{tabularx}

\begin{tabularx}{\linewidth}{@{}l X@{}}
    \textbf{Francés} & Intermedio \\
\end{tabularx}

\vfill
\center{\footnotesize Last updated: \today}

\end{document}
